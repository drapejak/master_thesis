\part{Závěr}
	V rámci této práce byl navrhnut, implementován a experimentálně vyzkoušen postup pro automatický odhad orientace faset broušených kamenů typu \textit{viva12} o rozměrech v jednotkách milimetrů. Vstupem do optimalizačního algoritmu je snímek s~obrazy svazků na stínítku, které vystupujících z~ozářeného kamene. Princip odhadu orientace faset spočívá v porovnání geometrie naměřených svazků s geometrií simulovaných svazků.
	
První část práce se zabývá detekcí světelných stop v obraze. Detekce spočívala ve zpracování výsledku detekce maximálně stabilních extrémních oblastí. Úspěšnost navržené detekce stop je srovnatelná s výsledkem detekce komerčně využívaného programu pro detekci hvězd na noční obloze. U pozorovaných svazků byl určen zářivý tok, směr šíření a detekovány ocásky vznikající v okolí obrazu svazku z důvodu neostrých hran kamene.

Druhá část práce byla zaměřena na hledání korespondencí simulovaných a pozorovaných svazků. Bylo ukázáno, že úloha korespondence svazků je obtížná. Byl navrhnut a implementován iterativní postup pro nalezení korespondujících svazků. 

Výsledky automatické optimalizace ukazují, že orientaci faset kamene lze podle navrženého postupu opakovaně odhadnout s přesností v řádech desetin úhlového stupně, zatímco pozorovaná výrobní chyby jsou v řádech stupňů.

Byla navržena metoda pro ověření správnosti korespondujících dvojic svazků.  Aplikovatelnost metody byla prozkoumána pomocí simulace v programu LADOK. Bylo ukázáno, že tato metoda funguje spolehlivě pouze pro korespondence vybraných tříd svazků. %Metoda spočívá v rozdělení korespondencí do čtyř disjunktních množin. První množina určuje model kamene, druhá množina slouží v výpočtu hodnotícího kritéria, třetí množina 

V této práci byl zároveň popsán nový příznak svazku definující směr a velikost úhlu rotace svazku při rotaci kamene. Znalost tohoto příznaku může zjednodušit úlohu korespondencí svazků zejména pro složitější tvary kamenů.\\

Tato práce přináší návod, jak postupovat při odhadu orientace faset broušených kamenů. Využití práce pro jiné tvary kamenů vyžaduje podrobný průzkum parametrů vystupujících svazků z ozářeného kamene.
	
	 
	  
	 
	   
	
	
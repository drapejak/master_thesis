  
\part{Výsledky}

Navržené algoritmy jsme použili pro automatickou optimalizaci kamenů tvaru \textit{viva12} podle algoritmu v kapitole \ref{sec: auto}. Vybrali jsme kameny se šesti odstíny. Nejčastěji je zastoupený odstín \textit{Crystal}, průhledné sklo. Soupis použitých kamenů je v tabulce \ref{tab:viva12desc}. 
 
\begin{table}[htps]
\centering
	\begin{tabular}{|C{1.2cm}|C{1.9cm}|C{1.2cm}|C{1.2cm}|C{1.2cm}|c||C{1.2cm}|C{1.2cm}|C{1.2cm}|C{1.2cm}|}
	\hline
	Kámen č. & Odstín 	& $d_{BOT}$ [\SI{}{\milli\metre}] 	& $h$ [\SI{}{\milli\metre}]  & $d_{TOP}$ [\SI{}{\milli\metre}] & $n$ & $\mathrm{E}(\sigma_{\alpha})$ [\SI{}{\degree}]  & $\mathrm{E}(\sigma_{\varepsilon})$ [\SI{}{\degree}]  & $\mathrm{E}(\delta_n)$ [\SI{}{\degree}]  & $\sigma(\delta_n)$ [\SI{}{\degree}] \\ \hline \hline
	1 & Hyacint	&	$2.90$			& $1.24$    &	$1.10$		&	3 & 0.089& 0.051 & 0.101 &0.071\\ \hline
	2 & Violet	&	$2.82$			& $1.16$	&	$1.15$		&	3 & 0.363& 0.237 & 0.451& 0.204\\ \hline
	3 & Citrine  &	$2.86$			& $1.14$	&	$1.20$		&	3 & 0.882& 0.598 & 1.092& 0.819\\ \hline
	4 & Hyacint	&	$2.90$			& $1.24$	&	$1.10$		&	3 & 0.172& 0.076 & 0.175& 0.108\\ \hline
	5 & Hyacint	&	$2.90$			& $1.28$	&	$1.10$		&	3 & 0.490& 0.356 & 0.561& 0.497\\ \hline
	6 & Hyacint	&	$2.88$			& $1.28$	&	$1.00$		&	4 & 0.269& 0.190 & 0.298& 0.248\\ \hline
	7 & Crystal	&	$2.88$			& $1.26$	&	$1.10$		&	3 & & & &\\ \hline
	8 & Crystal	&	$2.88$			& $1.26$	&	$1.15$		&	3 & & & &\\ \hline
	9 & Light Amethyst	& $2.82$    & $1.10$	&   $1.10$      &	3 & & & &\\ \hline % 21
	10 & Crystal	&	$4.80$		& $1.90$	&	$1.86$		&	3 & & & &\\ \hline % 22
	11 & Crystal	&	$4.78$		& $1.80$	&	$2.10$		&	3 & & & &\\ \hline % 23
	12 & Crystal	&	$3.96$		& $1.60$	&	$1.60$		&	4 & & & &\\ \hline % 24
	13 & Crystal	&	$4.00$		& $1.66$	&	$1.50$		&	3 & & & &\\ \hline % 25
	14 & Amethyst	&	$2.00$		& $1.00$	&	$0.80$		&	3 & & & &\\ \hline % 26
	15 & Crystal	&	$2.00$		& $0.94$	&	$0.80$		&	3 & & & &\\  \hline% 27
	16 & Crystal	&	$2.00$		& $0.92$	&	$0.80$		&	3 & & & &\\  \hline% 29
	
	\end{tabular}
	\caption{Popis rozměrů a barvy snímaných kamenů typu VIVA12 použitých při experimentech.}
	\label{tab:viva12desc}
\end{table}
	
	Všechny kameny jsem do měřicí soustavy umístili minimálně 3$\times$. Uložení kamene při dalším měření se lišilo v rotaci kamene kolem svislé osy.

	Abychom mohli porovnat výsledky optimalizace parametrů faset u jednotlivých kamenů, potřebujeme znát uložení kamene. Snažili jsem se o to, aby byly kameny otočeny přibližně o \SI{120}{\degree} v případě tří vzorků a \SI{90}{\degree} pro čtyři vzorky. Přesnou rotaci kamene však neznáme.
	
	Za ideálního stavu, kdy je orientace všech faset ideální, nejsme schopni dobře odhadnout vzájemnou rotaci uložení kamene při opakovaném měření. U použitých kamenů jsou fasety nerovnoměrně vychýleny od ideálního náklonu a vzájemnou rotaci kamene proto odhadneme, pokud docílíme maximální shody parametrů faset.   
	
	Výsledky automatické optimalizace kamene upravíme na shodnou rotaci kamene. 
	
	Normály faset \textbf{UF1}-\textbf{UF12} převedeme do azimutu a elevace. Pro jednotlivé fasety vypočítáme směrodatnou odchylku azimutů a elevací. Střední hodnota směrodatných odchylek azimutů je v tabulce \ref{tab:viva12desc} značena jako $\mathrm{E}(\sigma_{\alpha})$ a pro elevaci jako $\mathrm{E}(\sigma_{\varepsilon})$. Pro fasety \textbf{TOP} a \textbf{BOT} nemá cenu určovat azimut a elevaci, protože tyto normály jsou blízko singulárního bodu azimutu. 
	
	Pro každé dva výsledky optimalizace náklonu faset nalezneme úhel, který mezi sebou svírají normály faset. Pro $n$ měření fasety dostaneme $(n-1)!$ hodnot s velikostí úhlového rozdílu. Střední hodnota úhlových rozdílů pro všechny kombinace a pro všechny fasety je v tabulce \ref{tab:viva12desc} zapsána ve sloupci $\mathrm{E}(\delta_n)$ a směrodatná hodnota pro stejná data ve sloupci $\sigma(\delta_n)$. 

 \clearpage
  
\part{Výsledky}

\section{Vybrané kameny}

Navržené algoritmy jsme použili pro automatickou optimalizaci kamenů tvaru \textit{viva12} podle algoritmu z kapitoly \ref{sec: auto}. Vybrali jsme 14 kamenů pěti odstínů. Nejčastěji zastoupený odstín je \textit{Crystal}, průhledné sklo. Soupis použitých kamenů je v tabulce \ref{tab:viva12desc}. 

\section{Uložení broušených kamenů při měření}

 	Každý z kamenů jsme do měřicí soustavy umístili minimálně 3$\times$. Uložení kamene se při dalším měření  lišilo hlavně v rotaci kamene kolem svislé osy.

	Abychom mohli porovnat výsledky optimalizace parametrů faset u jednotlivých kamenů, potřebujeme znát uložení kamene. Snažili jsme se o to, aby byly kameny otočeny přibližně o~\SI{120}{\degree} v případě tří vzorků a \SI{90}{\degree} pro čtyři vzorky. Přesnou rotaci kamene však neznáme.
	
	Za ideálního stavu, kdy je kámen souměrný, nejsme schopni dobře odhadnout vzájemnou rotaci uložení kamene při opakovaném měření. U použitých kamenů jsou fasety náhodně vychýleny od ideálního náklonu a vzájemnou rotaci kamene proto odhadneme, pokud docílíme maximální shody sklonů faset.   
	
	Výsledky automatické optimalizace kamene upravíme na shodnou orientaci kamene. 
	
\section{Přesnost počátečního odhadu}
	
	Na výsledek optimalizace orientace faset má vliv počáteční odhad parametrů faset. Chyba počátečního odhadu orientace faset nesmí být příliš vysoká, abychom byli schopni nalézt korespondence základních tříd svazků (\textbf{1A}, \textbf{3A} a \textbf{5D}). Při optimalizaci sklonu faset podle základních tříd svazků poté optimalizační algoritmus nalezne zpravidla stejné lokální minimum nezávisle na počátečním odhadu orientace. 
	
	Na výsledek optimalizace má v dalším postupu zásadní vliv počáteční odhad vzdáleností faset od počátku souřadného systému kamene. Optimalizační algoritmus vzdálenost faset nemění. Chyba odhadu vzdáleností faset se projeví nejen ve velikosti zářivého toku simulovaných svazků, ale také v přítomnosti některých svazků, což může snadno vést k nesprávnému určení korespondujících svazků. Bohužel nemáme k dispozici přístroj, kterým bychom vzdá\-le\-nost faset změřili, proto nesmí být chyba počátečního odhadu vzdálenosti fasety od počátku sou\-řad\-né\-ho systému kamene příliš vysoká. 
	
	
\newpage
\section{Vyhodnocení přesnosti metody}

\paragraph{Chyby odhadu azimutu a elevace}
\hspace{1mm}

	Orientaci normál parametrizujeme za pomocí azimutu a elevace. Pro jednotlivé fasety UF1$-$UF12 vy\-po\-čí\-tá\-me směrodatnou odchylku azimutů a směrodatnou odchylku elevací. Střední hodnota směrodatných odchylek azimutů je v tabulce \ref{tab:viva12desc} značena jako $\mathrm{E}(\sigma_{\alpha})$ a pro elevaci jako $\mathrm{E}(\sigma_{\varepsilon})$. Pro fasety TOP a BOT nemá cenu určovat azimut a elevaci, protože tyto normály jsou blízko singulárního bodu. 

\paragraph{Směr normál}
\hspace{1mm}	

	Pro každé dva výsledky optimalizace sklonu fasety nalezneme úhel, který mezi sebou svírají normály hodnocené fasety. Pro $n$ měření fasety dostaneme $(n-1)!$ hodnot s velikostí úhlového rozdílu. Střední hodnota úhlových rozdílů pro všechny kombinace dané fasety a pro všechny fasety je v~tabulce \ref{tab:viva12desc} zapsána ve sloupci $\mathrm{E}(\delta_n)$. 
	
\paragraph{Chyba orientace vybroušených faset}
\hspace{1mm}	

Tento parametr hodnotí nepřesnost výroby jednotlivých kamenů. Uvádíme ji zde pro porovnání chyby měření a chyby výroby. U jednotlivých kamenů aplikujeme vektorový součet na všechny 3 resp. 4 výsledky normál fasety. Nalezneme úhel $\Delta\varphi$, který svírá vektor vzniklý vektorovým součtem s výkresovou normálou fasety. Střední hodnota úhlu $\Delta\varphi$ pro všech faset kamene je v tabulce \ref{tab:viva12desc} zapsána ve sloupci $\mathrm{E}(\Delta\varphi)$.

\begin{table}[htps]
\centering
	\begin{tabular}{|C{1.2cm}|C{1.9cm}|C{1.2cm}|C{1.2cm}|C{1.2cm}|c||C{1.2cm}|C{1.2cm}|C{1.2cm}|C{1.2cm}|}
	\hline
	Kámen č. & Odstín 	& $d_{BOT}$ [\SI{}{\milli\metre}] 	& $h$ [\SI{}{\milli\metre}]  & $d_{TOP}$ [\SI{}{\milli\metre}] & $n$ & $\mathrm{E}(\sigma_{\alpha})$ [\SI{}{\degree}]  & $\mathrm{E}(\sigma_{\varepsilon})$ [\SI{}{\degree}]  & $\mathrm{E}(\delta_n)$ [\SI{}{\degree}]  & $\mathrm{E}(\Delta\varphi)$ [\SI{}{\degree}] \\ \hline \hline
	1 & Hyacint		&	$2.90$		& $1.24$    &	$1.10$		&	3 & 0.16& 0.07 & 0.16 &1.39\\ \hline
	2 & Violet		&	$2.82$		& $1.16$	&	$1.15$		&	3 & 0.33& 0.16 & 0.36&3.72\\ \hline
	3 & Citrine  	&	$2.86$		& $1.14$	&	$1.20$		&	3 & 0.17& 0.12 & 0.23& 3.03\\ \hline
	4 & Hyacint		&	$2.90$		& $1.24$	&	$1.10$		&	3 & 0.19& 0.01 & 0.20& 5.10\\ \hline
	5 & Hyacint		&	$2.90$		& $1.28$	&	$1.10$		&	3 & 0.18& 0.14 & 0.24& 6.15\\ \hline
	6 & Hyacint		&	$2.88$		& $1.28$	&	$1.00$		&	4 & 0.20& 0.21 & 0.32& 1.01\\ \hline
	7 & Crystal		&	$2.88$		& $1.26$	&	$1.10$		&	3 & 0.23&0.38 &  0.48& 2.88\\ \hline
	8 & Crystal		&	$2.88$		& $1.26$	&	$1.15$		&	3 & 0.22&0.22 &0.33 &4.36\\ \hline
	9 & Light Amethyst	& $2.82$    & $1.10$	&   $1.10$      &	3 & 0.44& 0.31 & 0.53  & 3.53\\ \hline % 21
	10 & Crystal	&	$4.80$		& $1.90$	&	$1.86$		&	3 & 0.25 & 0.23 & 0.35 & 1.63\\ \hline % 22
	11 & Crystal	&	$4.78$		& $1.80$	&	$2.10$		&	3 & 0.36& 0.22 & 0.39&2.61\\ \hline % 23
	12 & Crystal	&	$3.96$		& $1.60$	&	$1.60$		&	4 & 0.39&0.32 & 0.48 & 1.48\\ \hline % 24
	13 & Crystal	&	$2.00$		& $0.94$	&	$0.80$		&	3 & 0.24&0.46 & 0.57& 3.13\\  \hline% 27
	14 & Crystal	&	$2.00$		& $0.92$	&	$0.80$		&	3 & 0.40 & 0.40 & 0.60 & 3.92\\ \hline  \hline% 29
	\multicolumn{6}{|l||}{Průměr} &0.27 & 0.23 &0.37 & 3.14 \\ \hline
	\end{tabular}
	\caption[Výsledek automatické optimalizace orientace faset.]{Popis rozměrů a barvy kamenů typu \textit{viva12} použitých při experimentech s výsledky automatické optimalizace orientace faset.}
	\label{tab:viva12desc}
\end{table}
	
	
	


 \clearpage
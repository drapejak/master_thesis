\section*{Abstrakt}
Tato práce se zabývá návrhem a kalibrací soustavy pro měření šíření laserového svazku šperkařskými kameny. Zkalibrovaná soustava bude použita k porovnání naměřených výsledků z měřeného kamene s jeho matematickým modelem. Porovnáním lze zjistit tvar kamene a kvalitu brusu, čehož se využívá např. pro určení ceny kamene nebo k seřizovaní brusných kotoučů při výrobě.   

Úkolem kalibrace navržené měřicí soustavy je nalezení správné transformace mezi pixely z pořízeného snímku a úhlem, pod kterým světelný paprsek opustil broušený šperk. Kalibrace probíhá postupně v několika krocích. Prvním krokem je nalezení projekční matice a radiálního zkreslení kamery. V dalším kroku s pomocí stereo-rekonstrukce hledáme parametry stínítka a nakonec optimalizujeme parametry laserového svazku. Pro optimalizaci použijeme nelineární metody minimalizace. 

\textbf{Klíčová slova:} kalibrace kamery, korespondence obrazů, triangulační metoda, optimalizace parametrů 

\section*{Abstract}
	This thesis describes the design and calibration of system for measuring the spread of the laser beam in the cut gemstone. Calibrated system will be used to compare results measured on the stone with its mathematical model. We can determine the shape and quality of the cut stone by comparing which is used e.g. to determine stone price or to adjust grinding wheel in the production.

The task of calibration is to find the correct transformation between the pixels in captured image and the direction of the light beam leaves the cut jewel. Calibration is carried out in several steps. The first step is to find the projection matrix and the radial distortion of the cameras. In the next step, we use stereo-reconstruction to find parameters of shade and then we optimize the parameters of the laser beam. We use nonlinear minimization methods to optimize.

\textbf{Keywords:} camera calibration, image correspondence, triangulation method, parameter optimization

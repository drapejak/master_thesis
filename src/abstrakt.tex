\section*{Abstrakt}
Tato práce se zabývá automatickým odhadem orientace faset broušeného kamene s využitím informace světelných svazcích vystupujících z ozářeného kamene. 

Vstupem do optimalizačního algoritmu je snímek s obrazy svazků na stínítku. Princip odhadu orientace faset spočívá v porovnání geometrie naměřených svazků s geometrií simulovaných svazků. Parametry simulovaného modelu kamene jsou upravovány gradientním optimalizačním algoritmem. 

 První část práce je zaměřena na detekci světelných stop v~obraze a výpočet parametrů odpovídajících svazků. Detekce stop je založena na přítomnosti maximálně stabilních extrémních oblastí (MSER) v~obraze. Druhá část obsahuje návrh algoritmů k nalezení korespondujících dvojic svazků.   

Výsledky jsou demonstrovány  na broušeném kamenu typu šatonová růže s~12 bočními fasetami.

\textbf{Klíčová slova:} počítačové vidění, MSER, sledování paprsků, hledání korespondencí, broušené kameny 

\section*{Abstract}

This thesis describes the automatic estimation of cut stone facet orientation with the use of information about the parallel light beam propagation.

The principle of the stone facet estimation is a comparison of simulated output beams geometry and measured beams geometry. Geometry is estimated from beam traces on the screen. Parameters of the~simulated model are estimated by the gradient optimization algorithm. 

The first part of thesis is focused on the light beams detection and calculating the~parameters of the corresponding beams. The~light beams detection is based on the~presence of the maximally stable extremal regions (MSER) in the image. The~second part contains the light beam matching algorithms.

The results are demonstrated on chaton rose with the twelve side facets.


\textbf{Keywords:} computer vision, MSER, ray-tracing, light beam matching, cut stones

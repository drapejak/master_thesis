\section*{Abstrakt}
Tato práce se zabývá automatickým odhadem orientace faset broušeného kamene s využitím informace o šíření rovnoběžného světelného svazku brusem. Důležité jsou světelné svazky vystupující z kamene. 

Metoda odhadu parametrů faset spočívá v porovnání svazků ze simulace a z měřicího experimentu. Parametry simulovaného modelu jsou upravovány gradientním optimalizačním algoritmem. Výstupem měřicího experimentu je snímek s obrazy svazků.

 První část je zaměřena na detekci světelných stop v obraze a výpočet parametrů odpovídajících svazků. Detekce stop je založena na přítomnosti maximálně stabilních extrémních oblastí (MSER) v obraze. Druhá část obsahuje návrh algoritmů k nalezení korespondujících dvojic svazků.   

Práce je zaměřena na broušený kámen typu šatonová růže s 12-ti bočními fasetami.

\textbf{Klíčová slova:} počítačové vidění, MSER, sledování paprsků, párování světelných svazků, broušené kameny 

\section*{Abstract}

This thesis describes the automatic estimation of cut stone facet orientation with the use of information about  the parallel light beam propagation. Light beams leaving the stone are significant for us.

The principle of stone facet estimation is a comparison of beams from the simulation and beams from the measurement experiment. Parameters of the simulated model are modified by the gradient optimization algorithm.  The image of light beams is the output of the measurement experiment. 

The first part is focused on light beams detection and calculating the parameters of the corresponding beams. Light beams detection is based on the presence of the maximally stable extremal regions (MSER) in the image. The second part contains the light beam matching algorithms.

The work is focused on chaton rose with the twelve side facets.


\textbf{Keywords:} computer vision, MSER, ray-tracing, light beam matching, cut stones

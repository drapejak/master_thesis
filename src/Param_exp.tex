\section*{Experimenty - závislost parametrů laserových svazků}

Cílem experimentů je získat jednotlivé funkce charakterizující závislosti parametrů detekovaných svazků $\kappa_i$ na referenčních parametrech $\psi_i$ určených matematickým modelem kamene. K parametrům budeme přistupovat jednotlivě, např. pro referenční zářivý tok se pokusíme nalézt co nejpřesnější aproximaci funkce, která by splňovala $\kappa_i = h_i\left(\psi_i\right)$.

Pro experimenty jsme použili korespondující svazky, které jsou výsledkem ruční optimalizace náklonu plošek kamene. Význam těchto korespondencí označujeme výrazem "\textit{ground truth}" podobně jako v oblasti strojového učení, kde slouží jako přesná data v trénovací množině pro určení statistických modelů. Použití výrazu "\textit{ground truth}" je ovšem v našem případě sporné, neboť určení množiny korespondencí závisí na osobě, která tato úlohu provádí. To, co v našem případě označujeme jako "\textit{ground truth}", je množina, o které se domníváme, že by měla obsahovat správné korespondence svazků, ale v našem případě nemůžeme s naprostou jistotou tvrdit, že tomu tak je. 

Ruční optimalizaci jsme provedli na broušených kamenech s označením VIVA12. Popis rozměrů a barev kamenů je v tabulce \ref{tab:viva12desc}. 

Broušený kámen ozářen laserovým svazkem o vlnové délce \SI{670}{\nano\metre}. Při průchodu světelného svazku kamenem se část záření absorbuje a přemění se na teplo. Absorpce záření je závislá na materiálu, ze kterého je materiál vyroben. To znamená, že bychom měli získat rozdílné funkce $\kappa_i = h_i_j\left(\psi_i\right)$ pro $j$ materiálů pro parametry laserových svazků, které závisí na velikosti zářivého toku. 

Předpokládáme, že kameny se stejným odstínem jsou vyrobeny ze stejně kvalitního materiálu a absorpce ve je v celém kameni konstantní, neboli kámen je perfektně čistý. Za těchto předpokladů můžeme spojit korespondence z více kamenů se stejným odstínem a získat věrohodnější odhad.    

\begin{table}[htps]
\centering
	\begin{tabular}{c|C{2cm}|C{2.2cm}|C{2cm}|C{2.3cm}|C{2cm}}
	Označení & Barva 	& Průměr spodku [\SI{}{\milli\metre}] 	& Výška [\SI{}{\milli\metre}]  & Průměr tabulky [\SI{}{\milli\metre}] & Počet snímků\\ \hline \hline
	viva\_11 & Hyacint	&	$2.90$			& $1.24$    &	$1.10$			&	3\\ \hline
	viva\_12 & Violet	&	$2.82$			& $1.16$	&	$1.15$			&	3\\ \hline
	viva\_13 & Citrine  &	$2.86$			& $1.14$	&	$1.20$			&	3\\ \hline
	viva\_14 & Hyacint	&	$2.90$			& $1.24$	&	$1.10$			&	3\\ \hline
	viva\_15 & Hyacint	&	$2.90$			& $1.28$	&	$1.10$			&	3\\ \hline
	viva\_16 & Hyacint	&	$2.88$			& $1.28$	&	$1.00$			&	4\\ \hline
	viva\_17 & Crystal	&	$2.88$			& $1.26$	&	$1.10$			&	3\\ \hline
%	viva\_18 & Crystal	&	$2.88$			& $1.26$	&	$1.15$			&	3\\ \hline
%	viva\_21 & Light Amethyst	& $2.82$    & $1.10$	&   $1.10$      	&	3\\ \hline
	viva\_22 & Crystal	&	$4.80$			& $1.90$	&	$1.86$			&	3\\ \hline
%	viva\_23 & Crystal	&	$4.78$			& $1.80$	&	$2.10$			&	3\\ \hline
%	viva\_24 & Crystal	&	$3.96$			& $1.60$	&	$1.60$			&	4\\ \hline
%	viva\_25 & Crystal	&	$4.00$			& $1.66$	&	$1.50$			&	3\\ \hline
%	viva\_26 & Amethyst	&	$2.00$			& $1.00$	&	$0.80$			&	3\\ \hline
	viva\_27 & Crystal	&	$2.00$			& $0.94$	&	$0.80$			&	3\\ 
	\end{tabular}
	\caption{Popis rozměrů a barvy snímaných kamenů typu VIVA12 použitých při experimentech.}
	\label{tab:viva12desc}
\end{table}

\section*{Zářivý tok}
Pokusíme se nalézt funkci $h_{flux}$, která by popisovala závislost referenčního zářivého toku svazků $\psi_{flux}$ na zářivého toku detekovaných svazků $\kappa_{flux}$, tak že $\psi_{flux} = h_{flux}\left( \kappa_{flux} \right))$. 

Nejprve si vykreslíme do grafu bodově příspěvky od jednotlivých korespondujících svazků ( obr. \ref{fig:fluxAll}) pro broušené kameny odstínu Hyacint. Tento graf však ukazuje, že funkci $h_{flux}$, kterou bychom mohli dostatečně charakterizovat závislost $\psi_{flux} = h_{flux}\left( \kappa_{flux} \right))$ nelze v těchto datech určit. Důvodů může být hned několik. 

Zaprvé víme, že v modelu kamene nejsou postihnuty všechny skutečnosti, které ovlivňují velikost zářivého toku stop. Můžeme zmínit například to, že v modelu není zahrnuta čistota materiálu, která nemusí a prakticky ani není v celém materiálu konstantní. Také nedokonalé leštění fasety ovlivňuje rozptylové parametry světelného svazku. Navíc malá změna náklonu fasety může vést k rapidnímu nárůstu plochy, která je zasažena daným svazkem a tím i zářivého toku, což si ukážeme v kapitole \ref{cap: fluxChange}. To je pouze část z dlouhého výčtu vlastností, které v konečném důsledku vyústí v chybu údaje na y-ové ose grafu 
$\psi_{flux} = h_{flux}\left( \kappa_{flux} \right))$.

Zadruhé musíme počítat s nejistotou určení zářivého toku detekovaných svazků vznikající při extrakci parametrů svazků z obrazu. Mezi zásadní problémy patří šum, překrývání stop, difuzní odrazivost svazků či odhad pozadí snímku.  


Ve grafu jsou patrné shluky bodů, které přesto, že mají referenční zářivý tok podobné velikosti se velikost detekovaného zářivého toku znatelně liší. Při detailnějším pohledu zjistíme, že se jedná o stopy třídy (XXX a XXX) s jedním respektive třemi optickými jevy (refrakce a reflexe) s fasetou (hity). To nás směřuje k myšlence rozdělit svazky podle počtu hitů. 

V naší množině korespondencí se objevují svazky s maximální počtem $11$-ti hitů. Svazky s počtem hitů $2$, $4$, $10$ a $11$ se v množině korespondencí prakticky nevyskytují. Svazky s $1$ resp. $3$ hity nám neposkytují příliš informace a výsledný graf vypadá spíše jako množina náhodných hodnot spadajících do vytočené oblasti.   





\section*{Velikost}

\section*{Intenzita}

\section*{Ocásky - velikost}

\section*{Ocásky - směr}

\section*{Elipsovitost?}

\section*{Azimut}

\section*{Elevace}








\subsection{Hladový algoritmus versus Maďarský algoritmus}


\subsection{Korespondence hlavních (nejlépe určitelných) párů}



\subsection{Párování podle ocásků}

\subsection{Párování podle pozice a zářivého toku }




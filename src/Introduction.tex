\part{Úvod a motivace}

Drahé kameny jsou lidstvem od pradávna vyhledávány.
Jsou symbolem krásy, bohatství a moci. Z počátku se kameny leštily do oblých tvarů a stávaly se součástí šperků.
S vývojem civilizace se začaly drahé kameny brousit, aby se zvýraznil lom světla a lesk minerálu. Broušením vznikaly rovinné fasety. Kombinací faset s definovanou velikostí a sklonem se vyvinuly standardní tvary jako brilliant, trilliant, rosette, baguette apod. Drahé a velmi cenné suroviny jako diamant se brousí pouze ručně. Méně cenné suroviny jsou převážně opracovávány automatickými stroji.  

Cenu broušených kamenů určují čtyři základní parametry. Mezi ně patří čistota, barva, hmotnost a kvalita brusu. 

Čistotu určuje množství příměsí v materiálu kamene a praskliny či vzduchové bubliny, které mohou vzniknout při tvorbě krystalu.

Broušené kameny dělíme do široké řady odstínů. Barva kamene závisí jeho chemickém složení. Důležitá je jednotnost barvy celého kamene.

Hmotnost souvisí s velikostí kamene. U diamantů se hmotnost určuje v karátech a je natolik důležitá, že se v některých případech volí kompromis mezi hmotností a kvalitou brusu.

Brus je důležitá mechanická úprava kamene. Ideálně opracovaný kámen má dokonalý soulad mezi brilancí, ohněm a jiskřením. Mezi parametry hodnotící kvalitu brusu patří kvalita povrchového opracování faset. Fasety se brousí rovinnými brusnými kotouči. Broušením mohou vzniknout rýhy, škrábance, prohlubně, abraze na hranách (zbroušení přechodu mezi fasetami), povrchová oxidace a nové fasety. Kámen je třeba brousit s vysokou přesností. Každá odchylka velikosti a úhlu fasety od ideálního tvaru zhorší optické vlastnosti kamene. Orientace faset je důležitým parametrem pro zhodnocení kvality brusu.

Jedním z nástrojů ke zkoumání optických vlastností broušeného kamene je nasvícení jeho povrchu světelným svazkem. Na fasetách broušeného kamene dochází ve zjednodušeném případě k odrazu a lomu světelného svazku. Z toho důvodu se dopadající svazek roztříští na řadu světelných svazků různých tvarů. Svazky vystupující z kamene jsou definovány směrem šíření, plochou průřezu, zářivým tokem a polarizací. Tyto svazky určují geometrii kamene. 

Tato práce navazuje na dlouhodobý výzkum v Centru strojového vnímání na katedře kybernetiky Elektrotechnické fakulty ČVUT. 

Cílem práce je detekovat svazky vycházející z kamene a navrhnout inverzní metodu, která z pozorovaných svazků určí geometrii kamene. Zaměříme se na broušené kameny o průměru v jednotkách milimetrů a složitějšího tvaru než čtverec.  

\clearpage

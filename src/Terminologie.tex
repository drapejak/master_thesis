\part{Terminologie}

Je důležité rozlišit svazky získané pomocí simulace v LADOKu a svazky získané experimentálním měřením.   
Proto zavedeme \textit{simulované} a \textit{pozorované} svazky.
\vspace{4mm}
\section{Simulované svazky}
\begin{enumerate}

\item 	Získáme potřebné parametry kamene, který zkoumáme. Zdrojem může být technický výkres, nebo předchozí měření.  

\item	Sestavíme model, který bude přibližně určovat tvar kamene. Tento model budeme považovat za \textit{referenční}.

\item	V programu LADOK simulujeme průlet svazku \textit{referenčním} modelem. Pro simulaci je důležité znát polarizaci zdrojového svazku a index lomu kamene. 

\item	Výsledkem simulace jsou parametry \textit{simulovaných} svazků.

\end{enumerate}

\section{Pozorované svazky}
Předpokladem pro získání parametrů pozorovaných svazků je sestavení a kalibrace měřicí soustavy podle \cite{Drapela}. % odkay na kapitolu
\begin{enumerate}

\item 	Opracovaný kámen umístíme do měřicí soustavy.

\item	Provedeme experiment průchodu svazku kamenem podobný situaci v simulačním programu LADOK. 

\item	Získáme obraz dopadu svazků na stínítko. 

\item	V obraze detekujeme světelné stopy (kapitola \ref{sec:detection}).  

\item	Z detekovaných stop vypočítáme parametry \textit{pozorovaných} svazků (kapitola \ref{sec:beam parameters}).

\end{enumerate}

\clearpage
%Přecházíme k situaci, kdy máme dostupné informace o \textit{simulovaných} i \textit{reálných} svazcích. Mezi těmito dvěma množinami je třeba nalézt korespondence. Korespondující svazky si odpovídají seznamem faset kamene, na které při své cestě dopadají.  
%
%Pro korespondující páry určíme chybovou funkci a parametry budeme optimalizovat. Optimalizační algoritmus odhadne takové nastavení parametrů, aby bylo optimalizované kritérium co nejmenší. 
%
%Optimalizované parametry použijeme k výpočtu \textit{optimalizovaného} modelu kamene. Orientace faset broušeného kamene odečteme z \textit{optimalizovaného} modelu.   